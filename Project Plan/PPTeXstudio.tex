\PassOptionsToPackage{dvipsnames}{xcolor}
\documentclass[a4paper,11pt]{report} %article

\usepackage{graphicx,subfigure,afterpage,hyperref,xspace,xcolor,caption,soul,geometry,pdfpages,stackengine,eso-pic,fancyhdr,hyphenat,listings,longtable,url,enumitem,fancyvrb,textcomp}


%\usepackage[utf8]{inputenc} %to make the single quote appear correct after the encoding inserted above!

%command to substitute "{\em MyTaxyService}" with "\mts"
\newcommand{\mts}{\mbox{\normalfont\itshape myTaxiService}}
\geometry{margin=1in}

%header & footer style
\fancyhead{}
\fancyhead[C]{\iffloatpage{}{\slshape\rightmark}}
\fancyfoot{}
\fancyfoot[C]{\iffloatpage{}{\thepage}}
\renewcommand{\headrulewidth}{\iffloatpage{0pt}{0.4pt}}
\renewcommand{\footrulewidth}{\iffloatpage{0pt}{0.4pt}}
\pagestyle{fancy}
\renewcommand{\sectionmark}[1]{\markright{\thesection.\ #1}}
\renewcommand{\subsectionmark}[1]{\markright{\thesubsection.\ #1}}

%tOC style: sections bold 
\usepackage[subfigure]{tocloft}
\renewcommand{\cftsecfont}{\bfseries}
\renewcommand{\cftsecpagefont}{\normalfont\bfseries}% page numbers in bold
\renewcommand{\cftdotsep}{1}
\renewcommand{\cftsecleader}{\bfseries\cftdotfill{\cftsecdotsep}}% dot leaders in bold

%to keep the links of the TOC invisible
\hypersetup{
	colorlinks,
	citecolor=black,
	filecolor=black,
	linkcolor=black,
	urlcolor=black
}


\title{Politecnico di Milano\\A.A. 2015/2016\\Software Engineering 2\\ \bigskip 
Assignment 5: Project Plan\\
{\normalsize Version 1.0}}
\author{Alessandro Baldassari (mat. 841561) \\ Alberto Bendin (mat. 841734) \\ Francesco Giarola (mat. 840554)}


%to set the nested bullet lists style
\renewcommand{\labelitemii}{$\circ$}
%\renewcommand{\labelitemii}{}
\renewcommand{\labelitemiii}{$\diamond$}

%to avoid the hyphenation of the name of the software
\hyphenation{myTaxyService}

\begin{document}
	
	%fIRSTPAGE
	
	%pOLIMI-LOGO
	\begin{figure}[t]
		\centering
		\includegraphics[width=1\linewidth]{"Pictures/polimi-logo"}
		\label{fig:polimi-logo}
	\end{figure}
	
	\maketitle
		
	
	%bLANK-PAGE
	\thispagestyle{empty}
	\clearpage\mbox{}\clearpage

	
	
	
	%to number the section from 1 instead of 0.1 with the report class, without using the article class. Avoid the forced use of chapters to number from 1. Tailored for REPORT class!!!
	\renewcommand*\thesection{\arabic{section}}
	\renewcommand*\thesubsection{\arabic{section}.\arabic{subsection}}
	\renewcommand*\thesubsubsection{%
	\arabic{section}.\arabic{subsection}.\arabic{subsubsection}%
	}
	\setcounter{secnumdepth}{4}
	\setcounter{tocdepth}{4}
	

	
	%to change the page numbering from roman in the toc to arabic
	\pagenumbering{roman}
	\tableofcontents
	\newpage
	\pagenumbering{arabic}
	
	
	%to insert the writing "Page" above page numbers in the TOC
	\addtocontents{toc}{~\hfill\textrm{Page}\par}
	
	%tables style
	\renewcommand{\arraystretch}{1.2}
	\setlength{\tabcolsep}{12pt}
	
	\section{Introduction} 
		\subsection{Purpose and Scope}
			The main purpose of the project plan is to plan time, cost and resources adequately to estimate the work needed and to effectively manage risk during project execution. A failure to adequately plan reduces the project's chances of successfully accomplishing its goals.\smallskip\\
			Project planning generally consists of:
			\begin{itemize}
				\item Identifying deliverables and creating the work breakdown structure;
				\item Identifying the activities needed to complete those deliverables and networking the activities in their logical sequence;
				\item Estimating the resource requirements for the activities;
				\item Estimating time and cost for activities;
				\item Developing the schedule;
				\item Developing the budget;
				\item Resource allocation (organization of work loads);
				\item Risk planning.
			\end{itemize} 
			This document is based on an analysis made with two different algorithmic metrics of the system for \mts{}. The first one is the Function Points (FP), which is used to estimate the software dimension (code size), which is directly used to evaluate the cost. The second is the COCOMO II that is used to estimate the efforts required in the development of a project by taking in account: characteristics of people, products and process. 
			
		\subsection{List of Definitions and Abbreviations}
			The following acronyms are used in this document:
			\begin{itemize}
				\item FP: Function Points
				\item COCOMO: COnstructive COst MOdel
			\end{itemize}
			The following definitions are used in this document:
			\begin{itemize}
				\item Deliverables: are work that are delivered to the customer, e.g. a requirement document for the system.
			\end{itemize}
			
	\section{Function Point analysis}
	
	\section{COCOMO II analysis}			
	
	\section{Tasks identification and schedule}	
	
	\section{References allocation}	
	
	\section{Risk planning and management}	

	\pagebreak
	\section{References}
		Material from Wikipedia
		\begin{itemize}
			\item Project management: \href{https://en.wikipedia.org/wiki/Project\_management\#Planning}{https://en.wikipedia.org/wiki/Project\_management\#Planning}
		\end{itemize}
	
	\section{Appendix}
		\subsection{Software and tools used}
		\begin{itemize}
			\item TeXstudio 2.10.6 (\href{http://www.texstudio.org/}{http://www.texstudio.org/}) to redact and format this document.
			\item Astah Professional 7.0 (\href{http://astah.net/editions/professional}{http://astah.net/editions/professional}) 
		\end{itemize}
		
		\subsection{Hours of work} The time spent to redact this document:
		\begin{itemize}
			\item Baldassari Alessandro: 12 hours.
			\item Bendin Alberto: 12 hours.
			\item Giarola Francesco: 12 hours.
		\end{itemize}

\end{document}